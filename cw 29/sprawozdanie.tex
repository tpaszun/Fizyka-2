\documentclass[a4paper,10pt,twoside]{article}
\usepackage[polish]{babel}
\usepackage[utf8]{inputenc}
\usepackage[T1]{fontenc}
\usepackage{indentfirst}
\usepackage[top=2.5cm, bottom=2.5cm, left=2.5cm, right=2.5cm]{geometry}
\usepackage{graphicx}
\usepackage{amsmath}
\usepackage{booktabs}

\begin{document}

\newcommand{\unit}[1]{\thinspace \mathrm{#1}}

\begin{center}
\bgroup
\def\arraystretch{1.5}
\begin{tabular}{|c|c|c|c|c|c|}
	\hline
	EAIiIB & \multicolumn{2}{|c|}{Piotr Morawiecki, Tymoteusz Paszun} & Rok II & {Grupa 3a} & {Zespół 6} \\
	\hline
	\multicolumn{3}{|c|}{\begin{tabular}{c}Temat: Fale podłużne w ciałach stałych \end{tabular}} &
	\multicolumn{3}{|c|}{\begin{tabular}{c}Numer ćwiczenia: 29 \end{tabular}} \\
	\hline
	\begin{tabular}{@{}c@{}}Data wykonania:\\8.11.2017r.\end{tabular} & \begin{tabular}{@{}c@{}}Data oddania:\\15.11.2017r.\end{tabular} &
	\begin{tabular}{c}Zwrot do poprawki:\\\phantom{data} \end{tabular} & \begin{tabular}{c}Data oddania:\\\phantom{data}\end{tabular} &
	\begin{tabular}{@{}c@{}}Data zaliczenia:\\\phantom{data}\end{tabular} & \begin{tabular}{c}Ocena:\\\phantom{ocena}\end{tabular} \\[4ex]
	\hline
\end{tabular}
\egroup
\end{center}


\section{Cel ćwiczenia}

Celem ćwiczenia jest wyznaczenie modułu Younga dla różnych materiałów na bazie pomiarów prędkości rozchodzenia się fal dźwiękowych (podłóżnych) w prętach.

\section{Wstęp teoretyczny}




$$ \lambda_i = \frac{2l}{i} $$

$$ v_i = \lambda_if $$

$$ E = \rho v^2 $$


\section{Wykonanie ćwiczenia}

\begin{itemize}
	\item Pomiary wymiarów próbek badanych materiałów.
	\item Pomiary masy próbek badanych materiałów.
	\item Pomiary częstotliwości dźwieku wydawanego przez pręty po uderzeniu.
\end{itemize}

\section{Wyniki pomiarów}

\subsection{Wymiary oraz masa próbek}


\begin{table}[!htbp]
\caption{Pomiary masy i wymiarów próbek badanych materiałów}
\centering
\def\arraystretch{1.4}
\begin{tabular}{@{}rcccccc@{}}
\\
\toprule
\begin{tabular}{@{}c@{}}Materiał\end{tabular} &
\begin{tabular}{@{}c@{}}Masa $\unit{[g]}$\end{tabular} &
\begin{tabular}{@{}c@{}}Wymiary $\unit{[mm]}$\end{tabular} &
\begin{tabular}{@{}c@{}}Objętość $\unit{[mm^3]}$\end{tabular} &
\begin{tabular}{@{}c@{}}Gęstość $\unit{[\frac{kg}{m^3}]}$\end{tabular} \\
\midrule
miedź  &  66  &  $d = 4,85,  l = 385 $  &  7112,69  &  9279,19 \\
stal  &  30,851  &  $ a = 14,15,  b = 14,25,  c = 19,8 $  &  3992,42  &  7727,39 \\
mosiądz  &  74  &  $ d = 6, l = 312 $  &  8821,59  &  8388,51 \\
aluminium  &  24  &  $ d = 5, l = 442 $  &   8678,65  &  2765,41 \\
\bottomrule
\end{tabular}
\end{table}


\subsection{Pręt miedziany}

Zmierzona długość pręta miedzianego: $ l = 1802 \unit{mm} $.


\begin{table}[!htbp]
\caption{Pomiary częstotliwości dla pręta miedzianego}
\centering
\def\arraystretch{1.4}
\begin{tabular}{@{}rcccccc@{}}
\\
\toprule
\begin{tabular}{@{}c@{}}Harmoniczna\end{tabular} &
\begin{tabular}{@{}c@{}}Częstotliwość \\ $\unit{[Hz]}$\end{tabular} &
\begin{tabular}{@{}c@{}}Delta pomiaru \\ częstotliwości $\unit{[Hz]}$\end{tabular} &
\begin{tabular}{@{}c@{}}Długość fali \\ $\unit{[mm]}$\end{tabular} &
\begin{tabular}{@{}c@{}}Prędkość fali \\ $\unit{[\frac{m}{s}]}$\end{tabular} \\
\midrule
$f_0$  &  $ \frac{1027,1 + 1031,8}{2} = 1029,45 $  &  4,81  &  3604,00  &  3710,14 \\
$f_1$  &  $ \frac{2059,7 + 2062,1}{2} = 2060,90 $  &  2,35  &  1802,00  &  3713,74 \\
$f_2$  &  $ \frac{3090,9 + 3094,4}{2} = 3092,65 $  &  3,53  &  1201,33  &  3715,30 \\
$f_3$  &  $ \frac{4121,7 + 4125,2}{2} = 4123,45 $  &  3,53  &   901,00  &  3715,23 \\
$f_4$  &  $ \frac{5154,0 + 5157,6}{2} = 5155,80 $  &  3,53  &   720,80  &  3716,30 \\
\midrule
       &           &        &  Średnia: &  3714,14 \\
\bottomrule
\end{tabular}
\end{table}


\subsection{Pręt stalowy}

Zmierzona długość pręta stalowego: $ l = 1802 \unit{mm} $.

\begin{table}[!htbp]
\caption{Pomiary częstotliwości dla pręta stalowego}
\centering
\def\arraystretch{1.4}
\begin{tabular}{@{}rcccccc@{}}
\\
\toprule
\begin{tabular}{@{}c@{}}Harmoniczna\end{tabular} &
\begin{tabular}{@{}c@{}}Częstotliwość \\ $\unit{[Hz]}$\end{tabular} &
\begin{tabular}{@{}c@{}}Delta pomiaru \\ częstotliwości $\unit{[Hz]}$\end{tabular} &
\begin{tabular}{@{}c@{}}Długość fali \\ $\unit{[mm]}$\end{tabular} &
\begin{tabular}{@{}c@{}}Prędkość fali \\ $\unit{[\frac{m}{s}]}$\end{tabular} \\
\midrule
$f_0$  &  $ \frac{1401,80 + 1407,70}{2} = 1404,75 $  &  5,88  &  3604,00  &  5062,72 \\
$f_1$  &  $ \frac{2903,80 + 2907,40}{2} = 2905,60 $  &  3,53  &  1802,00  &  5235,89 \\
$f_2$  &  $ \frac{4309,90 + 4314,60}{2} = 4312,25 $  &  4,71  &  1201,33  &  5180,45 \\
$f_3$  &  $ \frac{5715,40 + 5719,00}{2} = 5717,20 $  &  3,53  &   901,00  &  5151,20 \\
$f_4$  &  $ \frac{7123,30 + 7217,40}{2} = 7170,35 $  &  94,12  &   720,80  &  5168,38 \\
\midrule
       &           &        &  Średnia: &  5159,73 \\
\bottomrule
\end{tabular}
\end{table}

\subsection{Pręt z mosiądzu}

Zmierzona długość pręta wykonanego z mosiądzu: $ l = 998 \unit{mm} $.

\begin{table}[!htbp]
\caption{Pomiary częstotliwości dla pręta z mosiądzu}
\centering
\def\arraystretch{1.4}
\begin{tabular}{@{}rcccccc@{}}
\\
\toprule
\begin{tabular}{@{}c@{}}Harmoniczna\end{tabular} &
\begin{tabular}{@{}c@{}}Częstotliwość \\ $\unit{[Hz]}$\end{tabular} &
\begin{tabular}{@{}c@{}}Delta pomiaru \\ częstotliwości $\unit{[Hz]}$\end{tabular} &
\begin{tabular}{@{}c@{}}Długość fali \\ $\unit{[mm]}$\end{tabular} &
\begin{tabular}{@{}c@{}}Prędkość fali \\ $\unit{[\frac{m}{s}]}$\end{tabular} \\
\midrule
$f_0$  &  $ \frac{1679,80 + 1685,70}{2} = 1682,75 $  &  5,88  &  1996,00  &  3358,77 \\
$f_1$  &  $ \frac{3463,20 + 3472,10}{2} = 3467,65 $  &  8,82  &  998,00  &  3460,71 \\
$f_2$  &  $ \frac{5149,20 + 5161,00}{2} = 5155,10 $  &  11,76  &  665,33  &  3429,86 \\
$f_3$  &  $ \frac{6837,50 + 6940,40}{2} = 6888,95 $  &  102,94  &   499,00  &  3437,59 \\
$f_4$  &  $ \frac{8615,10 + 8629,80}{2} = 8622,45 $  &  14,71  &   399,20  &  3442,08 \\
\midrule
       &           &        &  Średnia: &  3425,80 \\
\bottomrule
\end{tabular}
\end{table}

\subsection{Pręt aluminiowy}

Zmierzona długość pręta wykonanego z aluminium: $ l = 1800 \unit{mm} $.

\begin{table}[!htbp]
\caption{Pomiary częstotliwości dla pręta aluminiowego}
\centering
\def\arraystretch{1.4}
\begin{tabular}{@{}rcccccc@{}}
\\
\toprule
\begin{tabular}{@{}c@{}}Harmoniczna\end{tabular} &
\begin{tabular}{@{}c@{}}Częstotliwość \\ $\unit{[Hz]}$\end{tabular} &
\begin{tabular}{@{}c@{}}Delta pomiaru \\ częstotliwości $\unit{[Hz]}$\end{tabular} &
\begin{tabular}{@{}c@{}}Długość fali \\ $\unit{[mm]}$\end{tabular} &
\begin{tabular}{@{}c@{}}Prędkość fali \\ $\unit{[\frac{m}{s}]}$\end{tabular} \\
\midrule
$f_0$  &  $ \frac{2422,10 + 2439,70}{2} = 2430,90 $  &  17,65  &  3600,00  &  8751,24 \\
$f_1$  &  $ \frac{4954,40 + 4972,10}{2} = 4963,25 $  &  17,65  &  1800,00  &  8933,85 \\
$f_2$  &  $ \frac{7389,70 + 7407,40}{2} = 7398,55 $  &  17,65  &  1200,00  &  8878,26 \\
$f_3$  &  $ \frac{9832,70 + 9926,80}{2} = 9879,75 $  &  94,12  &   900,00  &  8891,78 \\
$f_4$  &  $ \frac{11415,00 + 11432,00}{2} = 11423,50 $  &  17,65  &   450,00  &  8224,92 (wynik odstający)\\
\midrule
       &           &        &  Średnia: &  8863,78 \\
\bottomrule
\end{tabular}
\end{table}

\section{Wykresy}

\section{Opracowanie wyników}

\subsection{Analiza błędów}

\subsection{Niepewności pomiarów}


\subsection{Ocena zgodności uzyskanych wyników}

\section{Wnioski}

\end{document}
