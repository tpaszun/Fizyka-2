\documentclass[a4paper,10pt,twoside]{article}
\usepackage[polish]{babel}
\usepackage[utf8]{inputenc}
\usepackage[T1]{fontenc}
\usepackage{indentfirst}
\usepackage[top=2.5cm, bottom=2.5cm, left=2.5cm, right=2.5cm]{geometry}
\usepackage{graphicx}
\usepackage{amsmath}
\usepackage{booktabs}

\begin{document}

\newcommand{\unit}[1]{\thinspace \mathrm{#1}}

\begin{center}
\bgroup
\def\arraystretch{1.5}
\begin{tabular}{|c|c|c|c|c|c|}
	\hline
	EAIiIB & \multicolumn{2}{|c|}{Piotr Morawiecki, Tymoteusz Paszun} & Rok II & {Grupa 3a} & {Zespół 6} \\
	\hline
	\multicolumn{3}{|c|}{\begin{tabular}{c}Temat: Fale podłużne w ciałach stałych \end{tabular}} &
	\multicolumn{3}{|c|}{\begin{tabular}{c}Numer ćwiczenia: 29 \end{tabular}} \\
	\hline
	\begin{tabular}{@{}c@{}}Data wykonania:\\8.11.2017r.\end{tabular} & \begin{tabular}{@{}c@{}}Data oddania:\\15.11.2017r.\end{tabular} &
	\begin{tabular}{c}Zwrot do poprawki:\\\phantom{data} \end{tabular} & \begin{tabular}{c}Data oddania:\\\phantom{data}\end{tabular} &
	\begin{tabular}{@{}c@{}}Data zaliczenia:\\\phantom{data}\end{tabular} & \begin{tabular}{c}Ocena:\\\phantom{ocena}\end{tabular} \\[4ex]
	\hline
\end{tabular}
\egroup
\end{center}


\section{Cel ćwiczenia}

Celem ćwiczenia jest wyznaczenie modułu Younga dla różnych materiałów na bazie pomiarów prędkości rozchodzenia się fal dźwiękowych (podłóżnych) w prętach.

\section{Wstęp teoretyczny}




$$ \lambda_i = \frac{2l}{i} $$

$$ v_i = \lambda_if $$

$$ E = \rho v^2 $$


\section{Wykonanie ćwiczenia}

\begin{itemize}
	\item Pomiary wymiarów próbek badanych materiałów.
	\item Pomiary masy próbek badanych materiałów.
	\item Pomiary częstotliwości dźwieku wydawanego przez pręty po uderzeniu.
\end{itemize}

\section{Wyniki pomiarów}

\subsection{Wymiary oraz masa próbek}



\subsection{Pręt miedziany}

Zmierzona długość pręta: $ l = 1802 \unit{mm} $.

\begin{table}[!htbp]
\caption{Pomiary częstotliwości dla pręta miedzianego}
\centering
\def\arraystretch{1.4}
\begin{tabular}{@{}rcccccc@{}}
\\
\toprule
\begin{tabular}{@{}c@{}}Harmoniczna\end{tabular} &
\begin{tabular}{@{}c@{}}Częstotliwość \\ $\unit{[Hz]}$\end{tabular} &
\begin{tabular}{@{}c@{}}Delta pomiaru \\ $\unit{[Hz]}$\end{tabular} &
\begin{tabular}{@{}c@{}}Długość fali \\ $\unit{[mm]}$\end{tabular} &
\begin{tabular}{@{}c@{}}Prędkość fali \\ $\unit{[\frac{m}{s}]}$\end{tabular} &
\begin{tabular}{@{}c@{}}Moduł Younga \\ $\unit{[GPa]}$\end{tabular} \\
\midrule
$f_0$  &  1029,45  &  4,81  &  3604,00  & 3710,14 &  127,73 \\
$f_1$  &  2060,90  &  2,35  &  1802,00  &  3713,74 & 127,98 \\
$f_2$  &  3092,65  &  3,53  &  1201,33  &  3715,30 & 128,09 \\
$f_3$  &  4123,45  &  3,53  &   901,00  &  3715,23 & 128,08 \\
$f_4$  &  5155,80  &  3,53  &   720,80  &  3716,30 & 128,15 \\
\bottomrule
\end{tabular}
\end{table}

\subsection{Pręt stalowy}

\subsection{Pręt z mosiądzu}

\subsection{Pręt aluminiowy}

\section{Wykresy}

\section{Opracowanie wyników}

\subsection{Analiza błędów}

\subsection{Niepewności pomiarów}


\subsection{Ocena zgodności uzyskanych wyników}

\section{Wnioski}

\end{document}
