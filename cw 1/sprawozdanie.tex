\documentclass[a4paper,10pt,twoside]{article}
\usepackage[polish]{babel}
\usepackage[utf8]{inputenc}
\usepackage[T1]{fontenc}
\usepackage{indentfirst}
\usepackage[top=2.5cm, bottom=2.5cm, left=2.5cm, right=2.5cm]{geometry}
\usepackage{graphicx}
\usepackage{amsmath}
\usepackage{booktabs}

\begin{document}

\newcommand{\unit}[1]{\thinspace \mathrm{#1}}

\begin{center}
\bgroup
\def\arraystretch{1.5}
\begin{tabular}{|c|c|c|c|c|c|}
	\hline
	EAIiIB & \multicolumn{2}{|c|}{Piotr Morawiecki, Tymoteusz Paszun} & Rok II & {Grupa 3a} & {Zespół 6} \\
	\hline
	\multicolumn{3}{|c|}{\begin{tabular}{c}Temat: Wahadła fizyczne \end{tabular}} & 
	\multicolumn{3}{|c|}{\begin{tabular}{c}Numer ćwiczenia: 1 \end{tabular}} \\
	\hline
	\begin{tabular}{@{}c@{}}Data wykonania:\\26.10.2017r.\end{tabular} & \begin{tabular}{@{}c@{}}Data oddania:\\8.11.2017r.\end{tabular} & 
	\begin{tabular}{c}Zwrot do poprawki:\\\phantom{data} \end{tabular} & \begin{tabular}{c}Data oddania:\\\phantom{data}\end{tabular} &
	\begin{tabular}{@{}c@{}}Data zaliczenia:\\\phantom{data}\end{tabular} & \begin{tabular}{c}Ocena:\\\phantom{ocena}\end{tabular} \\[4ex]
	\hline
\end{tabular}
\egroup
\end{center}


\section{Cel ćwiczenia}

Celem ćwiczenia jest wyznaczenie momentu bezwładności brył sztywnych przez pomiar okresu drgań wahadła oraz na podstawie wymiarów geometrycznych. Badane bryły to pręt oraz pierścień.

\section{Wstęp teoretyczny}

\subsection{Wahadło fizyczne}

Wahadłem fizycznym nazywamy bryłę sztywną mogącą obracać się wokół osi obrotu $O$ nie przechodzącej przez środek masy $S$. Wahadło odchylone od pionu o kąt $\theta$, a następnie puszczone swobodnie będzie wykonywać drgania zwane ruchem wahadłowym. W ruchu tym mamy do czynienia z obrotem bryły sztywnej wokół osi $O$, opisuje go zatem druga zasada dynamiki dla ruchu obrotowego.
Zasada dynamiki dla ruchu obrotowego wyrażona jest wzorem $$I\varepsilon = M$$ gdzie $I$ - moment bezwładności, $\epsilon$ - przyspieszenie kątowe, $M$ - moment siły. Wartość przyspieszenia kątowego opisuje wzór $$\varepsilon = \frac{d^2\theta}{dt^2}$$

\subsection{Moment bezwładności na podstawie okresu drgań}
Dla wahadła fizycznego moment siły powstaje pod wpływem siły ciężkości. Dla wychylenia $\theta$ jest równy $$M = m g a \sin\theta$$ gdzie $a$ - odległość środka masy $S$ od osi obrotu $O$. Zatem równanie ruchu wahadła można zapisać jako $$ I_0 \frac{d^2\theta}{dt^2} = -m g a \sin \theta$$ gdzie $I_0$ - moment bezwładności względem osi obrotu przechodzącej przez punkt zawieszenia $O$. Jeżeli ograniczyć ruch do małych kątów wychylenia, to sinus kąta można zastąpić samym kątem w mierze łukowej, czyli $\sin\theta \approx \theta$. Przyjmując częstość określoną wzorem $\omega_0^2 = \frac{mga}{I_0}$  równanie ruchu przyjmuje postać równania oscylatora harmonicznego $$ \frac{d^2\theta}{dt^2}+\omega_0^2\theta(t) = 0$$. Okres drgań związany z częstością wynosi $$ T = 2 \pi \sqrt{\frac{I_0}{mga}} $$.

Przekształcając wzór otrzymujemy wzór na moment bezwładności $$ I_0 = (\frac{T}{2\pi})^2mga = \frac{mgaT^2}{4\pi^2}$$

\subsection{Moment bezwładności na podstawie prawa Steinera}

Dla wyznaczenia momentu bezwładności $I_S$ względem równoległej osi przechodzącej przez środek masy możemy posłużyć się związkiem między $I_0$ i $I_S$ znanym jako twierdzenie Steinera: $$ I_0 = I_S + ma^2$$

Wzór na moment bezwładności cienkiego pręta względem osi obrotu umieszczonej na końcu pręta to $$ I = \frac{1}{3}mL^2 $$ gdzie $L$ - długość pręta.

Wzór na moment bezwładności pierścienia względem osi obrotu przechodzącej przez jego środek to $$ I = \frac{1}{2}m(R^2 + r^2) $$ gdzie $R$ - zewnętrzny promień, $r$ - wewnętrzny promień.

\section{Opis doświadczenia}

\section{Wyniki pomiarów}

\begin{table}[!htbp]
\caption{Pomiary dla wahadła o długości $l=485 \unit{mm}$, czas mierzony co 20 okresów}
\centering
\def\arraystretch{1.4}
\begin{tabular}{@{}rcccc@{}}
\\
\toprule
\begin{tabular}{@{}c@{}}Lp.\end{tabular} &
\begin{tabular}{@{}c@{}}Liczba okresów $k$\end{tabular} &
\begin{tabular}{@{}c@{}}Czas $t$ dla $k$ okresów [$\unit{s}$]\end{tabular} &
\begin{tabular}{@{}c@{}}Czas $t'$ dla 20 okresów [$\unit{s}$]\end{tabular} &
\begin{tabular}{@{}c@{}}Czas 1 okresu [$\unit{s}$]\end{tabular}\\
\midrule
1  &  20   &  27,52   &  27,52  &  1,38  \\
2  &  40   &  62,36   &  34,84  &  1,74  \\
3  &  60   &  98,67   &  36,31  &  1,82  \\
4  &  80   &  133,45  &  34,78  &  1,74  \\
5  &  100  &  168,08  &  34,63  &  1,73  \\
6  &  120  &  202,58  &  34,50  &  1,73  \\
7  &  140  &  235,92  &  33,34  &  1,67  \\
8  &  160  &  275,92  &  40,00  &  2,00  \\
9  &  180  &  311,98  &  36,06  &  1,80  \\
10 &  200  &  349,08  &  37,10  &  1,86  \\
\bottomrule
\end{tabular}
\end{table}


\begin{table}[!htbp]
\caption{Pomiary dla wahadła o długości $l=485 \unit{mm}$, czas mierzony co 30 okresów}
\centering
\def\arraystretch{1.4}
\begin{tabular}{@{}rcccc@{}}
\\
\toprule
\begin{tabular}{@{}c@{}}Lp.\end{tabular} &
\begin{tabular}{@{}c@{}}Liczba okresów $k$\end{tabular} &
\begin{tabular}{@{}c@{}}Czas $t$ dla $k$ okresów [$\unit{s}$]\end{tabular} &
\begin{tabular}{@{}c@{}}Czas $t'$ dla 30 okresów [$\unit{s}$]\end{tabular} &
\begin{tabular}{@{}c@{}}Czas 1 okresu [$\unit{s}$]\end{tabular}\\
\midrule
1  &  30   &  40,11   &  40,11  &  1,34  \\
2  &  60   &  90,39   &  50,28  &  1,68  \\
3  &  90   &  144,45  &  54,06  &  1,80  \\
4  &  120  &  193,17  &  48,72  &  1,62  \\
5  &  150  &  245,76  &  52,59  &  1,75  \\
\bottomrule
\end{tabular}
\end{table}


\begin{table}[!htbp]
\caption{Pomiary dla zmiennej długości wahadła}
\centering
\def\arraystretch{1.4}
\begin{tabular}{@{}rccrr@{}}
\\
\toprule
\begin{tabular}{@{}c@{}}Długość wahadła [$\unit{mm}$]\end{tabular} &
\begin{tabular}{@{}c@{}}Czas 20 okresów [$\unit{s}$]\end{tabular} &
\begin{tabular}{@{}c@{}}Czas 1 okresu [$\unit{s}$]\end{tabular} &
\begin{tabular}{@{}c@{}}Wartość $g$ [$\unit{\frac{m}{s^2}}$]\end{tabular} \\
\midrule

135  &  14,23  &  0,71  &  10,53  \\
175  &  16,10  &  0,81   &  10,66  \\
215  &  18,40  &  0,92    &  10,03  \\
255  &  19,09  &  0,96  &  11,05  \\
295  &  20,56  &  1,03   &  11,02  \\
335  &  23,00  &  1,15    &  10,00  \\
375  &  24,81  &  1,24  &  9,62  \\
415  &  25,59  &  1,28  &  10,00  \\
455  &  26,75  &  1,34  &  10,04  \\
485  &  27,73  &  1,39  &  9,96   \\

\bottomrule
\end{tabular}
\end{table}

\newpage

\section{Opracowanie wyników}
\section{Wnioski}


\end{document}
