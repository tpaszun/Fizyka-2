\documentclass[a4paper,10pt,twoside]{article}
\usepackage[polish]{babel}
\usepackage[utf8]{inputenc}
\usepackage[T1]{fontenc}
\usepackage{indentfirst}
\usepackage[top=2.5cm, bottom=2.5cm, left=2.5cm, right=2.5cm]{geometry}
\usepackage{graphicx}
\usepackage{amsmath}
\usepackage{booktabs}

\begin{document}

\newcommand{\unit}[1]{\thinspace \mathrm{#1}}

\begin{center}
\bgroup
\def\arraystretch{1.5}
\begin{tabular}{|c|c|c|c|c|c|}
	\hline
	EAIiIB & \multicolumn{2}{|c|}{Piotr Morawiecki, Tymoteusz Paszun} & Rok II & {Grupa 3a} & {Zespół 6} \\
	\hline
	\multicolumn{3}{|c|}{\begin{tabular}{c}Temat: Wahadła fizyczne \end{tabular}} & 
	\multicolumn{3}{|c|}{\begin{tabular}{c}Numer ćwiczenia: 0 \end{tabular}} \\
	\hline
	\begin{tabular}{@{}c@{}}Data wykonania:\\26.10.2017r.\end{tabular} & \begin{tabular}{@{}c@{}}Data oddania:\\8.11.2017r.\end{tabular} & 
	\begin{tabular}{c}Zwrot do poprawki:\\\phantom{data} \end{tabular} & \begin{tabular}{c}Data oddania:\\\phantom{data}\end{tabular} &
	\begin{tabular}{@{}c@{}}Data zaliczenia:\\\phantom{data}\end{tabular} & \begin{tabular}{c}Ocena:\\\phantom{ocena}\end{tabular} \\[4ex]
	\hline
\end{tabular}
\egroup
\end{center}


\section{Cel ćwiczenia}

Celem ćwiczenia jest wyznaczenie momentu bezwładności
brył sztywnych przez pomiar okresu drgań wahadła oraz na podstawie wymiarów
geometrycznych. 

\section{Wstęp teoretyczny}

\subsection{Wahadło fizyczne}

Wahadłem fizycznym nazywamy bryłę sztywną mogącą obracać się wokół osi obrotu $O$ nie przechodzącej przez środek masy $S$. Wahadło odchylone od pionu o kąt $\theta$, a następnie puszczone swobodnie będzie wykonywać drgania zwane ruchem wahadłowym. W ruchu tym mamy do czynienia z obrotem bryły sztywnej wokół osi $O$, opisuje go zatem druga zasada dynamiki dla ruchu obrotowego.
Zasada dynamiki dla ruchu obrotowego wyrażona jest wzorem $$I\varepsilon = M$$ gdzie $I$ - moment bezwładności, $\epsilon$ - przyspieszenie kątowe, $M$ - moment siły. Wartość przyspieszenia kątowego opisuje wzór $$\varepsilon = \frac{d^2\theta}{dt^2}$$

\subsection{Moment bezwładności na podstawie okresu drgań}
Dla wahadła fizycznego moment siły powstaje pod wpływem siły ciężkości. Dla wychylenia $\theta$ jest równy $$M = m g a \sin\theta$$ gdzie $a$ - odległość środka masy $S$ od osi obrotu $O$. Zatem równanie ruchu wahadła można zapisać jako $$ I_0 \frac{d^2\theta}{dt^2} = -m g a \sin \theta$$ gdzie $I_0$ - moment bezwładności względem osi obrotu przechodzącej przez punkt zawieszenia $O$. Jeżeli ograniczyć ruch do małych kątów wychylenia, to sinus kąta można zastąpić samym kątem w mierze łukowej, czyli $\sin\theta \approx \theta$. Przyjmując częstość określoną wzorem $\omega_0^2 = \frac{mga}{I_0}$  równanie ruchu przyjmuje postać równania oscylatora harmonicznego $$ \frac{d^2\theta}{dt^2}+\omega_0^2\theta(t) = 0$$. Okres drgań związany z częstością wynosi $$ T = 2 \pi \sqrt{\frac{I_0}{mga}} $$.

Przekształcając wzór otrzymujemy wzór na moment bezwładności $$ I_0 = (\frac{T}{2\pi})^2mga = \frac{mgaT^2}{4\pi^2}$$

\subsection{Moment bezwładności na podstawie prawa Steinera}

Dla wyznaczenia momentu bezwładności $I_S$ względem równoległej osi przechodzącej przez środek masy możemy posłużyć się związkiem między $I_0$ i $I_S$ znanym jako twierdzenie Steinera: 
$$ I_0 = I_S + ma^2$$

Wzór na moment bezwładności cienkiego pręta względem osi obrotu umieszczonej na końcu pręta to $$ I = \frac{1}{3}mL^2 $$ gdzie $L$ - długość pręta.

Wzór na moment bezwładności pierścienia względem osi obrotu przechodzącej przez jego środek to $$ I = \frac{1}{2}m(R^2 + r^2) $$ gdzie $R$ - zewnętrzny promień, $r$ - wewnętrzny promień.

\section{Opis doświadczenia}

\section{Wyniki pomiarów}
\subsection{Pomiary masy i długości}
\begin{table}[!htbp]
	\caption{Pomiary masy i długości dla pretu}
	\centering
	\def\arraystretch{1.4}
	\begin{tabular}{@{}rcc@{}}
		\toprule
		\begin{tabular}{@{}c@{}}\end{tabular} &
		\begin{tabular}{@{}c@{}}Wartość  \end{tabular} &
		\begin{tabular}{@{}c@{}}Niepewność \end{tabular}\\
		\midrule
		$m$ [$\unit{g}$]  &  658   &  1    \\
		$l$ [$\unit{mm}$]  &  738   &  1    \\
		$b$ [$\unit{mm}$]  &  99  &  1    \\
		$a$ [$\unit{mm}$]  &  270  &  1    \\
		\bottomrule
	\end{tabular}
\end{table}
\begin{table}[!htbp]
	\caption{Pomiary masy i długości dla pierścienia}
	\centering
	\def\arraystretch{1.4}
	\begin{tabular}{@{}rcc@{}}
		\toprule
		\begin{tabular}{@{}c@{}}\end{tabular} &
		\begin{tabular}{@{}c@{}}Wartość  \end{tabular} &
		\begin{tabular}{@{}c@{}}Niepewność \end{tabular}\\
		\midrule
		$m$ [$\unit{g}$]  &  1360   &  1    \\
		$D_w$ [$\unit{mm}$]  &  249   &  1    \\
		$D_z$ [$\unit{mm}$]  &  279  &  1    \\
		$R_w$ [$\unit{mm}$]  &  124,5  &  1    \\
		$R_z$ [$\unit{mm}$]  &  139,5  &  1    \\
		$e$ [$\unit{mm}$]  &  9,7  &  0,05    \\
		$a$ [$\unit{mm}$]  &  129,8  &  0,05    \\
		\bottomrule
	\end{tabular}
\end{table}




\subsection{Pomiary okresu drgań}
\begin{table}[!htbp]
\caption{Pomiary okresu drgań dla prętu}
\centering
\def\arraystretch{1.4}
\begin{tabular}{@{}rccc@{}}
\toprule
\begin{tabular}{@{}c@{}}Lp.\end{tabular} &
\begin{tabular}{@{}c@{}}Liczba okresów $k$\end{tabular} &
\begin{tabular}{@{}c@{}}Czas $t$ dla $k$ okresów [$\unit{s}$]\end{tabular} &
\begin{tabular}{@{}c@{}}Czas 1 okresu [$\unit{s}$]\end{tabular}\\
\midrule
1  &  30  &  39,72   &    1,324  \\
2  &  30  &  39,61   &    1,320  \\
3  &  30  &  39,58   &    1,319  \\
4  &  30  &  39,66   &    1,322  \\
5  &  30  &  39,48   &    1,316  \\
6  &  30  &  39,60   &    1,320  \\
7  &  30  &  39,46   &    1,315  \\
8  &  30  &  39,33   &    1,311  \\
9  &  50  &  65,68   &    1,314  \\
10 &  50  &  65,75   &    1,315  \\
\midrule
   &  &    Wartość średnia okresu $T$: 1,318&\\
\midrule
   &  &   Niepewność $u(T)$: 0,000015 & \\
\bottomrule
\end{tabular}
\end{table}
\begin{table}[!htbp]
	\caption{Pomiary okresu drgań dla pierścienia}
	\centering
	\def\arraystretch{1.4}
	\begin{tabular}{@{}rccc@{}}
		\\
		\toprule
		\begin{tabular}{@{}c@{}}Lp.\end{tabular} &
		\begin{tabular}{@{}c@{}}Liczba okresów $k$\end{tabular} &
		\begin{tabular}{@{}c@{}}Czas $t$ dla $k$ okresów [$\unit{s}$]\end{tabular} &
		\begin{tabular}{@{}c@{}}Czas 1 okresu [$\unit{s}$]\end{tabular}\\
		\midrule
		1  &  30  &  31,04   &    1,035  \\
		2  &  30  &  30,83   &    1,028  \\
		3  &  30  &  31,01   &    1,034  \\
		4  &  30  &  31,05   &    1,035  \\
		5  &  30  &  31,12   &    1,037  \\
		6  &  30  &  30,96   &    1,032  \\
		7  &  30  &  30,91   &    1,030  \\
		8  &  30  &  31,16   &    1,039  \\
		9  &  30  &  31,17   &    1,039  \\
		10 &  30  &  30,86   &    1,029  \\
		\midrule
		&  &    Wartość średnia okresu $T$: 1,034&\\
		\midrule
		&  &   Niepewność $u(T)$: 0,000014 & \\
		\bottomrule
	\end{tabular}
\end{table}





\newpage

\section{Opracowanie wyników}
\section{Wnioski}


\end{document}
