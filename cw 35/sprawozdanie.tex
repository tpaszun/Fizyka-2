\documentclass[a4paper,10pt,twoside]{article}
\usepackage[polish]{babel}
\usepackage[utf8]{inputenc}
\usepackage[T1]{fontenc}
\usepackage{indentfirst}
\usepackage[top=2.5cm, bottom=2.5cm, left=2.5cm, right=2.5cm]{geometry}
\usepackage{graphicx}
\usepackage{amsmath}
\usepackage{booktabs}

\begin{document}

\newcommand{\unit}[1]{\thinspace \mathrm{#1}}

\begin{center}
\bgroup
\def\arraystretch{1.5}
\begin{tabular}{|c|c|c|c|c|c|}
	\hline
	EAIiIB & \multicolumn{2}{|c|}{Piotr Morawiecki, Tymoteusz Paszun} & Rok II & {Grupa 3a} & {Zespół 6} \\
	\hline
	\multicolumn{3}{|c|}{\begin{tabular}{c}Temat: Mostek Wheatstone'a \end{tabular}} &
	\multicolumn{3}{|c|}{\begin{tabular}{c}Numer ćwiczenia: 35 \end{tabular}} \\
	\hline
	\begin{tabular}{@{}c@{}}Data wykonania:\\22.11.2017r.\end{tabular} & \begin{tabular}{@{}c@{}}Data oddania:\\29.11.2017r.\end{tabular} &
	\begin{tabular}{c}Zwrot do poprawki:\\\phantom{data} \end{tabular} & \begin{tabular}{c}Data oddania:\\\phantom{data}\end{tabular} &
	\begin{tabular}{@{}c@{}}Data zaliczenia:\\\phantom{data}\end{tabular} & \begin{tabular}{c}Ocena:\\\phantom{ocena}\end{tabular} \\[4ex]
	\hline
\end{tabular}
\egroup
\end{center}


\section{Cel ćwiczenia}

\section{Wstęp teoretyczny}


\section{Wykonanie ćwiczenia}


\section{Wyniki pomiarów}

\section{Wykresy}

\section{Opracowanie wyników}

\subsection{Analiza błędów}

\subsection{Niepewności pomiarów}

\subsection{Ocena zgodności uzyskanych wyników}

\section{Wnioski}

\end{document}
